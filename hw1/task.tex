\section{Основное задание}

Это домашнее задание является вводным, в нем вам необходимо настроить
последовательный порт, контроллер прерываний и интервальный таймер.

\begin{enumerate}
  \item Инициализировать контроллер последовательного порта и создать функцию
        записи в последовательный порт. Мы будем использовать последовательный
        порт вместо экрана.
  \item Настроить программируемый контроллер прерываний i8259~\cite{INTEL:8259}
        (такой раньше использовался в персональных компьютерах, но сейчас его не
        используют, ему на замену пришел APIC, который поддерживает обратную
        совместимость).
  \item Настроить программируемый интервальный таймер~\cite{INTEL:8253} на
        прерывания через равные интервалы времени. В обработчике прерывания вам
        нужно выводить на последовательный порт сообщение (текст сообщения не
        существенен, но перед тем как расходиться, помните, что у проверяющего
        может отсутствовать чувство юмора и шутка может плохо закончится).
\end{enumerate}

\section{Дополнительные задания}

\begin{itemize}
  \item В целях отладки очень часто полезно видеть backtrace, т. е. как
        программа пришла к той или иной строчке кода. Задание заключается в том
        чтобы написать функцию, которая выводит backtrace в последовательный
        порт в виде набора адресов. Используйте эту функцию в обработчике
        исключений. При этом запрещается инструментировать функции, т. е.
        нельзя модифицировать функции, чтобы они самостоятельно добавляли себя
        в backtrace. Backtrace не должен быть совершенно точным. Построение
        backtrace не должно приводить к ошибкам (будет странно, если при выводе
        сообщения об ошибке при выводе backtrace-а вы полезите не в свою память,        что в свою очередь может привести к другой ошибке).
  \item Реализуйте функции семейства *printf (а именно printf, vprintf,
        snprintf и vsnprintf). Функция должна поддерживать как минимум
        следующие спецификаторы формата: d, i, u, o, x, c, s, p, и следующие
        модификаторы размера: hh, h, l, ll, z~\cite{CPP:PRINTF}. При реализации
        нельзя ограничивать размер вывода одного вызова printf/vprintf каким-то
        фиксированным размером, т. е. нельзя просто реализовать \{v\}snprintf и
        вызвать ее из \{v\}printf передав буффер фиксированного размера. 
\end{itemize}
