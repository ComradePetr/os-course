\begin{frame}
\frametitle{Правила и рекомендации}
\framesubtitle{Сначала рекомендации}

\begin{itemize}
  \item<1-> Не откладывайте домашние задание на последний день\footnote{Далее
            почему очень-очень не рекомендуется так делать.}
  \item<2-> Задавайте вопросы:
    \begin{itemize}
      \item<3-> не задаете вопросов - значит все понимаете!
      \item<4-> пришел на пару - задал вопрос! \onslide<5->{Нет смысла ходить
                на пары если вы все и так знаете.}
    \end{itemize}
  \item<6-> Не забивайте на задания:
    \begin{itemize}
      \item<7-> я помню, что я сказал, что одно задание можно полностью завалить
      \item<8-> задания не совсем независимы, так что так или иначе вам придется
                сделать большую часть из них
    \end{itemize}
\end{itemize}
\end{frame}


\begin{frame}
\frametitle{Правила и рекомендации}
\framesubtitle{Теперь правила}

\begin{itemize}
  \item<1-> Дедлайны
    \begin{itemize}
      \item<2-> их нужно соблюдать!
      \item<3-> их можно обсуждать...\footnote{Не значит, что я их перенесу.}
    \end{itemize}
  \item<4-> Списывание (с известных источников или одногруппников)
    \begin{itemize}
      \item<5-> не отслеживаем намеренно и не наказываем специально
      \item<6-> но и не поощряем; \onslide<7->{мы будем задавать вопросы, если
                вы на них не отвечаете - вы получаете 0.}
    \end{itemize}
  \item<8-> Дополнительные задания
    \begin{itemize}
      \item<9-> не принимаются после дедлайна (как и любые другие задания)
      \item<10-> не принимаются если не сделаны все основные (если не сказано
                 обратное)
    \end{itemize}
\end{itemize}
\end{frame}
