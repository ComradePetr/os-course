\begin{frame}
\frametitle{60-ые}

В 60-ых годах появились:
\begin{itemize}
  \item Compatible\footnote{интересно с чем?} Time-Sharing System (1962) -
        первая ОС с вытесняющей многозадачностью
  \item MCP (Master Control Program, кто смотрел Трон?) (1963) - первая ОC
        написанная на языке высокого уровня (на каком-то диалекте ALGOL-а);
  \item IBM System/360 (1964) - ОС система старалась удовлетворить нуждам всех,
        поэтому была сложной, стоимость ее разработки была заоблачной, а ошибки
        содержавшиеся в ней неисчислимы; но зато Фредерик Брукс написал про это
        книгу (я уверен вы все про нее слушали, а если нет, то еще услышите)
\end{itemize}
\end{frame}

\begin{frame}
\frametitle{60-ые}

А также:
\begin{itemize}
  \item первые миникомпьютеры (1961, фирма DEC выпустила PDP-1, а потом и другие
        из этой серии, в том числе и PDP-7, для которой был написан первый Unix)
  \item первая компьютерная мышь :)
\end{itemize}
\end{frame}

\begin{frame}
\frametitle{60-ые}
\framesubtitle{Дисковые устройства}

\begin{itemize}
  \item первые дисковые устройства появились в 1956 (емкость в 3.75 Mb, что
        больше даже дискет 3'25) - они выдавались в аренду на месяц :)
  \item в 60-ых дисковые устройства становятся массовыми, и появилась целое
        семейство дисковых ОС (<какой-нибудь префикс кроме MS> DOS)
  \item появление дисков - появление файловых систем!
\end{itemize}
\end{frame}
