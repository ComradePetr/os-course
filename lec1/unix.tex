\begin{frame}
\frametitle{MULTICS, 1964-?}

MULTICS был амбициозным проектом (слишком амбициозным):
\begin{itemize}
  \item вся память отображена на диск и доступна через файловую систему
  \item динамическая компоновка
  \item IPC через разделяемую память
  \item поддержка многопроцессорности
  \item апргрейд оборудования системы без перезагрузки (!!!)
  \item мудреная система защиты
  \item иерархическая файловая система
  \item командный интерпретатор в userspace
  \item pipe-ы
\end{itemize}
\end{frame}

\begin{frame}
\frametitle{Первый UNIX}

Пока академики развлекались, все инженеры бросили MULTICS и сделали
\href{http://www.cs.berkeley.edu/~brewer/cs262/unix.pdf}{UNIX} - маленькую и
простую ОС, зато рабочую:

\begin{block}{The UNIX Time Sharing System}
Perhaps the most important achievement of UNIX is to demonstrate that a
powerful operating system for interactive use need not be expensive either in
equipment or in human effort: UNIX can run on hardware costing as little as
\$40,000, and less than two man years were spent on the main system software.
\end{block}
\end{frame}
