\begin{frame}
\frametitle{Ранняя история компьютеров}
\framesubtitle{ENIAC, 1945}

Первый электронный компьютер общего назначения - ENIAC\footnote{Первенство очень
спорное, но в любом случае это была большая-большая куча хлама}:
\begin{itemize}
  \item память 200 десятичных знаков
  \item ввод/вывод 8000 десятичных знаков в минут
  \item 5000 операций в секунду
  \item программировался набором переключателей (перфолентой)
  \item полный провал с UX (какая там ОС, тут даже пару чисел сложит проблема)
\end{itemize}

\end{frame}

\begin{frame}
\frametitle{Ранняя история компьютеров}
\framesubtitle{Компьютеры с памятью}

Далее идет поколение компьютеров обладавших памятью, их было много, ОС все еще
не видно на горизонте:
\begin{itemize}
  \item EDSAC, EDVAC, 1949 - кто придумывал им названия?
  \item BINAC, 1949 - первый компьютер с интерпретатором (поверьте вы бы не
        захотели таким пользоваться, см.~\href{https://en.wikipedia.org/wiki/Short_Code_(computer_language)}{Short Code})
  \item UNIVAC, 1951 - нормальное переиспользование кода только появляется и
        это приближает нас к первым ОС
\end{itemize}

\end{frame}

\begin{frame}
\frametitle{IBM 701, 1952}
\framesubtitle{Первая "ОС"...}

\begin{itemize}
  \item IBM 701 - первый коммерческий компьютер (читай серийный)
  \item как и BINAC был с интерпретатором, который назывался IBM SpeedCoding
        System, который занимал треть всей памяти :)
  \item SHARE (Society to Help Alleviate Redundant Effort) - репозиторий
        подпрограмм, например, для общения с внешними устройствами (этакие
        первые драйвера устройств)
\end{itemize}

\end{frame}

\begin{frame}
\frametitle{Прерывания, 1956}

Впервые прерывания появились в одной из версий UNIVAC в 1956 году. При получении
прерывания компьютер сохранял указатель команд по специальному адресу в памяти.

Чтобы вернуться из прерывания нужно было сделать переход по адресу сохраненному
в этой специальной ячейке памяти.
\end{frame}

\begin{frame}
\frametitle{Пакетная обработка}

Самая долгая операция на первых процессорах - загрузка программы. Почему? Потому
что ее делал человек:
\begin{itemize}
  \item компьютер мог запускать только одну программу обрабатывать ее от начала
        и до конца и на этом все
  \item человек загружал программу в память компьютера из каких-нибудь перфокарт
  \item когда программа завершалась, человек снимал результаты работы программы
        (например, дамп памяти), убирал перфокарты, забирал распечатки и тд.
\end{itemize}
\end{frame}

\begin{frame}
\frametitle{Пакетная обработка}

Главная задача первых ОС увеличить пропускную способность и уменьшить простой
дорого компьютера устранив, на сколько это возможно человека.

Для этого программы снабжались описанием (используемые ресурсы, ожидаемое время
работы и тд) и загружались пачкой в систему, а дальше она сама по описанию
разбиралась что с ними делать.
\end{frame}
